\documentclass[12pt]{article}
\usepackage[utf8]{inputenc} % Кодировка utf8
\usepackage[T2A]{fontenc}
\usepackage{pscyr} % Нормальные шрифты
\usepackage{cmap} % Улучшенный поиск русских слов в полученном pdf-файле
\usepackage{amssymb}
\usepackage{textcomp}
\usepackage[english, russian]{babel} % Языки: русский, английский
\usepackage{amsmath}
\usepackage{graphicx} %графика
\begin{document}
\begin{center}
	\textbf{Введение}
\end{center}
\begin{center}
	\textbf{Галава 1}
\end{center}
	
В этой главе мы рассмотрим те особые методы, которые будем использовать для обработки исходных медицинских изображений, содержащие участок паталогии, определенный для лучевой терапии. Обозначим цели, проведём классификацию,подробно опишем каждый алгоритм, а также выясним, почему выбрали тот, а не иной метод для данных снимков.\\

Начнём с того, что определим важность предобработки изображений для дальнейшей успешной работы с ними. 
Первым шагом является получение цифрового изображения с использованием датчиков в оптических или тепловых длинах волн. Двумерное изображение, которое регистрируется этими датчиками, является отображением трехмерного визуального мира. Захваченные двумерные сигналы дискретизируются и квантуются для получения цифровых изображений.
Довольно часто мы получаем изображения, содержащие в себе разнообразные искажения, которые получаются из-за какого-то механизма ухудшения качества и различных помех. Такие искажения принято называть \textit{шумом}.\\
Выделяют следующие источники шума:\\
\begin{itemize}
	\item некачественное оборудование для захвата изображения (видеокамера, ТВ-тюнер, сканер и т.д.)
	\item плохие условия съёмки (ночная фото/видеосъемка, дождь,..)
	\item помехи, возникающие при передаче по аналоговым каналам, т.е. собственные шумы активных компонентов (усилителей) линии передачи
	\item повреждение информации на носителе или искажение данных при их передаче через цифровые каналы
	\item неточности при выделении яркостного и цветоразностных сигналов из аналогового композитного сигнала и т.п.
	\item энергетические помехи из-за беспорядочно распределённых отражателей сигнала, слишком мелких для того, чтобы их могла отобразить система 
\end{itemize}
Приведём основную классификацию шума.\\
\begin{enumerate}
	\item По способу искажения:\\
	\begin{itemize}
		\item аддитивный шум
		\item мультипликативный шум
		\item импульсный шум
		\item спекл-шум
		\item гауссов (нормальный) шум
		\item шум квантования
	\end{itemize}
	\item С точки зрения визуального восприятия:\\
	\begin{itemize}
		\item белый шум 
		\item цветные пятна
		\item биение пикселей
		\item шум, вызываемый помехами электросети
		\item вертикальные царапины
	\end{itemize}
\end{enumerate}

В некоторых из вышеперечисленных случаев нам нужны соответствующие методы улучшения снимков, чтобы полученные изображения имели лучшее визуальное качество, не содержали аберраций и шумов. Наличие шумов затрудняет дальнейшую успешную работу со снимками, и, из-за раличного характера и происхождения шумов, для каждого из них нужно определять подходящий алгоритм их устранения.\\

Рассмотрим подробнее и опишем именно те шумы и искажения, что имеются в исследуемых нами изображениях.\\ 
\textbf{Определение 1.1} \textit{"Виньетирование"} (фр. vignette — заставка) — феномен неполного ограничения (затемнения) наклонных пучков света оправой или диафрагмами оптической системы. В результате получаем снижение яркости снимка к краям поля зрения системы (то есть к углам кадра имеется значительное затемнение, в сравнении с серединой изображения). С большей вероятностью виньетирование появляется в широкоугольных объективах, телеобъективах, а также в оптике с большой светосилой. При уменьшении относительного отверстия (диафрагмы) в оптической системе эффект виньетирования снижается или пропадает вовсе. На широкоугольных объективах виньетирование может появляться в случае использования светофильтров, поэтому для широкоугольных линз рекомендуется покупать фильтры с тонкой оправой.\\

В исследуемых нами изображениях можно обнаружить данное явление виньетирования. \textbf{\textit{Рассмотрим один из исследуемых снимков "9945"}}. На данном снимке можно заметить, что край значительно затемнён, словно был применён радиальный градиент. Присутствие виньетки значительно затрудняет работу по обнаружению участков патологии и отделению от этого участка здоровой зоны. Если рассмотреть преобразование изображения, содержащего такое затемнение, из цветного в бинарное, то мы обнаружим, что, помимо проблемной зоны, затемнена будет и зона, где имеется явление виньетки. \textbf{\textit{Данное явление можно увидеть на изображении 1.2.}} Нежелательное виньетирование можно частично или полностью убрать при редактировании фотографий, с помощью подбора подходящего алгоритма.\\

Перейдём к основным понятиям.\\
\textbf{Определение 1.1} \textit{"Сегментация"} - это процесс, при котором изображение делится на ряд однородных областей. Каждая однородная область является составной частью или объектом всей сцены. Другими словами, сегментация изображения определяется набором областей, которые связаны и не перекрываются, так что каждый пиксель в сегменте изображения получает уникальную метку области, которая указывает, к какой области он принадлежит. Сегментация - один из важнейших элементов
в автоматическом анализе изображений, главным образом потому, что на этом этапе объекты или другие представляющие интерес объекты извлекаются из изображения для последующей обработки, такой как описание и распознавание. 
\begin{center}
	\textbf{Глава 2}
\end{center}
\begin{center}
	\textbf{Заключение}
\end{center}
\end{document}	